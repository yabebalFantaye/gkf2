% 2010/09/24: DemoBibTeX.tex is a very simple tutorial
% to list all the inputs in a bibliography database
% Created by Alain Coulais, LERMA, Observatoire de Paris
%Adapted by Rodrigo Leonardi, ESAC/ESA

%\documentclass[longauth,structabstract,referee]{aa}

\documentclass[longauth,traditabstract,referee]{aa}

\usepackage{color}
\usepackage{showkeys}
\usepackage{txfonts} % A&A font !!
\usepackage{natbib}
\bibpunct{(}{)}{;}{a}{}{,} % to follow the A&A style
\usepackage{url}
\input Planck.tex

\oddsidemargin=2.1cm
\evensidemargin=2.1cm

\begin{document}

\title{The \textit{Planck} Bibliography v5.9}

%
\author{%\fontsize{6pt}{6pt}\selectfont
        Planck Collaboration~\thanks{Corresponding author; email: planck\_publication\_management@rssd.esa.int}
       }
%
\def\planckinstitute#1{\gdef\@institute{#1}}
\def\institutename{\par
 \begingroup
 \parindent=0pt
 \parskip=0pt
 \setcounter{inst}{1}%
 \def\and{\par\stepcounter{inst}%
 \hangindent=25pt
 \noindent
 \hbox to\instindent{\hss\tiny${\theinst}$\enspace}\ignorespaces}%
 \setbox0=\vbox{\def\thanks##1{}\@institute}
 \ifnum\value{inst}>99\relax\setbox0=\hbox{${888}$\enspace}%
   \else\ifnum\value{inst}>9\relax\setbox0=\hbox{${88}$\enspace}%
   \else\setbox0=\hbox{${8}$\enspace}\fi\fi
 \instindent\wd0\relax
 \ifnum\value{inst}=1\relax
 \else
   \setcounter{inst}{1}%
   \hangindent=25pt
   \noindent
   \hbox to\instindent{\hss\tiny${\theinst}$\enspace}\ignorespaces
 \fi
 \tiny
 \ignorespaces
 \@institute\par
 \endgroup}
%
\planckinstitute{}
\providecommand{\sorthelp}[1]{}

%\date{Received \ \ ; accepted \ \ }

\abstract {This is a short guide to help \Planck\ authors to cite the papers
currently included in the \Planck\ Bib\TeX\ database.  A human-readable list of
the the articles currently included in the database is also provided.}

\keywords{Bibliography}

\authorrunning{Leonardi, Coulais, and Catalano}
\titlerunning{The \Planck\ Bib\TeX\ database}

\maketitle

\section{The \Planck\ bibliographical database}

The \Planck\ bibliographical database consists of one file called
\begin{verbatim} Planck_bib.bib, \end{verbatim} which contains references,
commonly used in papers by
the Planck Collaboration, in a format that is standardized and
readable by La\TeX\ compilers. An additional file,
\begin{verbatim} Planck_bib.tex,
\end{verbatim} produces this description and a
human-readable version of the contents of the database.  The reference data
were extracted from the NASA's Astrophysics Data System
(\verb=http://adswww.harvard.edu/=).

\section{Where to get it}

The \verb=Planck_bib.bib= file is
distributed from the Planck Publication Management Portal (PMP), at
\begin{verbatim}www.rssd.esa.int/index.php?project=PLANCK&page=Repositories.\end{verbatim}

\section{How to use it}

Place the file \verb=Planck_bib.bib= in a
directory accessible to your La\TeX\ compiler.  To add a new bibliographic
entry in the La\TeX\ source file, just add
\begin{verbatim}
\cite{Label}
\end{verbatim}
with ``Label'' being one of the labels in the left column of the list at the
end of this guide.  To compile the \TeX\ file, you just need to run the
following procedure:
\begin{verbatim}
$ pdflatex MyFile
$ bibtex MyFile
$ pdflatex MyFile
$ pdflatex
MyFile
\end{verbatim}

The bibliographic style of the (e.g., A\&A) journal is
automatically used through the insertion of
\verb=\bibliographystyle{aa}=.  Unless you use new entries, you don't need to
re-run the call to Bib\TeX. You can see very easily which entry is missing in
the log files.  One advantage of such a bibliographical database is
that you don't need to care about the number of entries.  Extra
entries are just skipped by Bib\TeX\ at compilation time.

\section{Extending it}

The \verb=Planck_bib.bib= file should
contain all references in \Planck\ papers which are used by more than
one paper.  Please submit your favorite references in this category
by sending an email to
\begin{verbatim}planck_publication_management@rssd.esa.int,\end{verbatim}
including in the
text of the email the full entries to the reference(s) in Bib\TeX\
format (they are automatically generated by the ADS search engine).
The Bib\TeX\ file will be updated and a new one placed in the PMP,
hopefully within a few days of your request.

\section{Full list of references in human-readable format}

To see the full list of entries in a Bib\TeX\ file (or more than one),
just insert
\begin{verbatim}
\nocite{*}
\end{verbatim}
in a La\TeX\ file.  This is what we do to display below the human-readable list
of current entries in the Planck database. The list displays the citation
label, the reference in the A\&A style (as provided by the \verb=aa.bst= file),
and the title of the article (the title actually doesn't show up in the A\&A
style, and it appears in this list only for easier identification of the
article during the writing process).
% here we say we would like to see all entries
\nocite{*}

% we load the A&A style for the biblio (''aa.bst'' file)
\bibliographystyle{aa}

% we load the bilbiographic file containing the entries
\bibliography{Planck_bib}

% we load the A&A style for the biblio (''aa.bst'' file)
%\bibliographystyle{plainnat}

% we load the bilbiographic file containing the entries
%\bibliography{Planck_bib}

\end{document}
