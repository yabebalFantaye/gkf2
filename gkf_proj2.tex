%\documentclass{article}

%,twocolumn

\documentclass[aps,prd,showpacs,superscriptaddress,groupedaddress]{revtex4-1}  % for review and submission
%\documentclass[aps,preprint,showpacs,superscriptaddress,groupedaddress]{revtex4}  % for double-spaced preprint
\usepackage{graphicx}  % needed for figures
\usepackage{dcolumn}   % needed for some tables
\usepackage{bm}        % for math

%%%%%%%%%%%%%%%%%%%%%%%%%%%%%%%%%%%%%%%%%%%%%%%%%%%%%%%%%%%%%%%%%%%%%%%%%%%%%%%%%%%%%%%%%%%%%%%%%%%%%%%%%%%%%%%%%%%%%%%%%%%%
\usepackage{amsfonts}
\input{journals_list.txt}

\usepackage{amssymb,latexsym,amssymb,amsmath,amsbsy,amsopn,amstext,upgreek}
\usepackage{color}
\usepackage{graphicx,wrapfig,fancybox,watermark,graphics}
\usepackage{pgf}
\usepackage{float}

\usepackage{vmargin}
\usepackage{cjhebrew}
\usepackage{subfigure}
\usepackage[toc,page]{appendix}


% avoids incorrect hyphenation, added Nov/08 by SSR
\hyphenation{ALPGEN}
\hyphenation{EVTGEN}
\hyphenation{PYTHIA}


\newcommand{\homeLinux}{/home/yabebal/}
\newcommand{\homeMac}{/Users/yabebal/}
\newcommand{\home}{\homeLinux}

\newcommand{\figdir}[1]{figures/#1}


\def\rms{root mean square~}
\newcommand{\figref}[1]{Figure (\ref{#1})}
%\newcommand{\eqref}[1]{Eqn. (\ref{#1})}
\newcommand{\reffig}[1]{Figure (\ref{#1})}
\newcommand{\refeq}[1]{Eqn. (\ref{#1})}
\newcommand{\refsec}[1]{Section \ref{#1}}
\newcommand{\secref}[1]{Section \ref{#1}}
\newcommand{\healpix}{HEALpix~}
\newcommand{\nside}[1]{N_{\rm side}}


\begin{document}

\title{A New Geometric Approach for High-Frequency Spherical Data Analysis\thanks{%
Research supported by ERC Grant 277742 Pascal.}}
\author{Yabebal Fantaye}
\affiliation{Stellenbosch....}
\author{Domenico Marinucci}
\affiliation{Dipartimento di Matematica, Universit\'{a} di Roma
''Tor Vergata'', Via della Ricerca Scientifica 1, I-00133 Roma,
Italy}
\author{Valentina Cammarota}
\affiliation{Department of Mathematics, King's College London, United Kingdom}


%{Authors}

\date{\today}



\begin{abstract}
The use of Minkowski Functionals as a probe for CMB data has long been widespread; in this context, the Gaussian Kinematic Formula (Adler and Taylor, 2007) allows analytic prediction of expected values under a variety of circumstances. More recently, sharp analytic predictions on the variances of Minkowski functionals when evaluated on the spherical harmonic components of an isotropic random field have also been given; likewise, analytic expressions for the expected value and the variance of critical points on these same components have also been established very recently. In this paper, we exploit these results to present a number of new tests for the presence of residual point sources or other contaminants, under the assumption that full-sky maps can be studied. An extended numerical experiment shows how these tests are very sensitive to the presence of residual contaminants.
\end{abstract}

\keywords{CMB, Data Analysis, Minkowski Functionals,
Gaussian Kinematic Formula, Spherical Harmonics}

\pacs{98.80.Es, 95.75.Mn, 95.75.Pq, 02.50.-r}

% 98.80.Es     Observational cosmology
% 95.75.Mn     Image processing
% 95.75.Pq     Mathematical procedures and computer techniques
% 02.50.-r Statistics

\maketitle

\section{Introduction}

The interplay between stochastic geometry tools and Cosmological data analysis has proved to be extremely fruitful in the last decade or so. More precisely, the geometrical properties of excursion sets for random fields of cosmological interest have been investigated in a huge amount of different theoretical and applied papers. It is well-known that these geometrical properties can be characterized in terms of the so-called Minkowski functionals, which in the two dimensional case correspond to the excursion area, (half) the boundary length and the Euler-Poincar\'e characteristic (connected regions minus number of holes) of the excursion set. Seminal contributions in the application of Minkowski functionals to astrophysical data were given by \cite{tomita1986_genus,
Coles1988_mfCMB}; many other papers have followed including for instance \cite{schmalzing1997_mf,
  dolgov1999_mfpolm, naselsky1998_mfPol, Novikov2000_mfCMB,
  schmalzing1998_mfreview,matsubara2003_mf2PT,
  wmap2003_wmf,wmap2007_wmf,starck2007}, or more recently \cite{natoli2010mfBoomrang, matsubara2010_mfFnl,
  ducout2013_mfFnl, gratten2012_mfLSSreview, munshi2013_mfSkewCl,planck2013_IS}. In most of these cases, Minkowski functionals have been applied to the original spherical data; more recently, however, applications to wavelet/needlet components have also been considered by the Planck collaboration \cite{planck2014-a18}.

%Excursizing

Geometrical and topological quantities of a field are defined on a set of objects generated by the excursion of the original field. For any real number $u$, the excursion sets of a random field $T(x)$ are defined by%
\begin{equation}
A_{u}(T):=\left\{ x:T(x) \geq u\right\} \text{ ,}
\end{equation}
or, more explicitly, they represent the subregions where the value of $T$ exceeds the threshold $u$. As mentioned earlier, in the  two-dimensional case the three Minkowski functionals are the area,$\emph{M}_{0}$, the boundary length, $\emph{M}_{1}$, and the Euler-Poincar\'{e} characteristic (number of connected components minus holes), $\emph{M}_{0}$; these functionals can be computed on real data by means of
accurate and numerically efficient algorithms \cite{KLENK2006127, Guderlei2007, Gay2012_NGPeaks}. 

A lot of mathematical efforts have been spent since the '80s on the characterization of expected values of these functionals under Gaussianity, culminating in the discovery of the beautiful Gaussian Kinematic Formula \cite{TaylorAdler2009}; comparing these expected values with realization allows the implementation of a number of tests for Gaussianity and Isotropy (see again \cite{planck2013-p09, planck2014-a18}.

While the behaviour of expected values is now fully understood, it is clear that the implementation of more sophisticated, hence more sensitive, testing procedures requires further knowledge, in particular the variance of these functionals and therefore the possibility to establish Central Limit Theorems with correct normalization factors. Establishing a Central Limit result requires of course the exploitation of a suitable notion of asymptotic behaviour; in the framework of spherical fields, the only relevant notion seems to be the one of High-Frequency asymptotics. In particular it is well-known that isotropic random fields on the sphere can be decomposed by means of the Spectral representation theorem into the sum of orthogonal components, each of them corresponding to a different multipole $\ell$. Very recently, precise formulae for the variance of Minkowski functionals evaluated on the excursion sets of each of these components were derived in \cite{cm1603}.


Our aim in this paper is to exploit these results to propose some new tests aimed at the detection of residual point sources or other foreground contaminants on spherical random field. As such, our procedures are naturally connected to the investigation of CMB data; the latter can be viewed as realizations of Gaussian and isotropic spherical random fields, which provide, in a loose sense, a snapshot of the Universe at the Big Bang time (or more precisely, at the so-called Recombination Era, which is now reckoned to have occurred some $3 \times 10^5$ years after the Big Bang. High resolution CMB maps have been very recently provided by the Planck Collaboration \cite{planck2013-p01}; they are commonly recognized as a goldmine of Cosmological information, and as such they have drawn an enormous amount of attention in the physical community. 

A major statistical challenge when investigating such maps is the removal of "foreground" contaminants, i.e. astrophysical objects which "mask" CMB radiation. The testing procedures we are going to advocate are meant to the detection of residual point sources which have not been removed from the CMB map; while this framework provides a natural environment for applications, we stress that many other fields where spherical data are of interest could also be considered. 

Minkowski Functionals are not the only objects of interest in this paper. Indeed, some other recent contributions have derived neat analytic formulae for the expected number and the variance of critical points on the same spherical harmonics components as considered earlier for MFs \cite{cm1603}. In this paper, we are exploiting these results for the first time as a further tool to search for residual point sources.   


The plan of the paper is as follows: in Section 2, we illustrate some background material on the expected value and variance of MFs evaluated on spherical harmonic components under the simplest conditions, e.g., full-sky Gaussian maps. In Section 3 we present the exact expected values of local peaks and their variances, and demonstrate their validity numerically. In Section 4 we present our detailed numerical studies for the implementation of these tools as a testing probe. Section 5 draws some conclusions and presents directions for future work.

\section{Characterization of Excursion Sets for Random Spherical Harmonics}

In the case of the two-dimensional sphere, the excursion sets $A_{u}(f)$ of a given (possibly random)
function $f$ are defined as
\begin{equation}
A_{u}(f):=\left\{ x\in S^{2}:f(x)\geq u\right\} \text{ .}
\end{equation}
Of course, in the limit where we take $u=-\infty ,$ we have that $%
A_{u}(f)=S^{2}$.

In this paper we aim at the implementation of Minkowski functionals/Lipschitz-Killing curvatures (to be defined below) on the spherical harmonic/Fourier components of of observed data. Let us first recall the well-known spectral representation theorem for spherical random fields, which states that the following identity holds, in the $L^"$ sense:
\begin{equation}
f(x)=\sum_{\ell =1}^{L_{\max }}\sum_{m=-\ell }^{\ell }a_{\ell m}Y_{\ell
m}(x)=\sum_{\ell =1}^{L_{\max }}f_{\ell }(x)\text{ ;}  \label{specrap}
\end{equation}%
The spherical harmonic coefficients may be computed from the field $f(.)$ by means of the inverse transform

\begin{equation}
a_{\ell m}= \int_{S^2} f(x) \bar{Y}_{\ell
m}(x)dx, \text{ } \text{ } \ell =1,2,...., m=-\ell,...,\ell ,   \label{specrap2}
\end{equation}%

The inverse transform \eqref{specrap2} is only feasible for unmasked (full-sky) data, a condition which is usually considered rather difficult to meet for astrophysical experiments such as those concerning CMB. Rather recently, however, full-sky maps were produced for instance by \cite{starketal2014} and by the Planck collaboration in its 2018 release (see .....). 

%\begin{equation*}
%A_{I}(f_{\ell }; {\cal S}^2):=\left\{ x\in S^{2}:f_{\ell }(x)\in I \right\}.
%\end{equation*}
%For $I=[u,\infty)$, we have 
%\begin{equation*}
%A_{u}(f_{\ell }; {\cal S}^2):=\left\{ x\in S^{2}:f_{\ell }(x) \ge u \right\}.
%\end{equation*}


Let us now recall again the definitions of the \emph{Lipschitz-Killing Curvatures} (LKCs), which correspond to Minkowski functionals up to a different indexing and normalization factors; in two dimensions, they are given by  (a) the Euler-Poincar\'{e} characteristic (written $\mathcal{L}_{0}(A_{u}(f))$), e.g. the number of connected
regions minus the number of holes, or two minus the genus; (b) half the boundary length of
the excursion regions (written $\mathcal{L}_{1}(A_{u}(f))$); the area of the excursion
regions (written $\mathcal{L}_{2}(A_{u}(f))$), which corresponds to the first Minkowski functional. The expected values of these functionals when evaluated on the excursion sets of Gaussian fields has been fully characterized by the Gaussian Kinematic Formula (GKF), see \cite{TaylorAdler2009}. 

We now need the family of functions $\rho _{l}(u)$ defined as
\begin{equation}
\rho _{l}(u)=(2\pi )^{-(l+1)/2}H_{l-1}(u)e^{-u^{2}/2}\text{ ,}
\end{equation} where $H_{k}(u)$ denotes as usual the family of Hermite polynomials, that is,
\begin{equation}
H_{0}(u)=1, H_{1}(u)=u, H_{2}(u)=u^{2}-1;
\end{equation}
it is convenient to define also
\begin{equation}
H_{-1}(u)=\sqrt{2\pi }(1-\Phi (u))e^{u^{2}/2},
\end{equation}
where $\Phi (u)$ is the Gaussian cumulative distribution function, whence%
\begin{eqnarray}
\rho _{0}(u) &=&(2\pi )^{-1/2}\sqrt{2\pi }(1-\Phi
(u))e^{u^{2}/2}e^{-u^{2}/2}=(1-\Phi (u)) \\
\rho _{1}(u) &=&\frac{1}{2\pi }e^{-u^{2}/2}\text{ , }\rho _{2}(u)=\frac{1}{%
\sqrt{(2\pi )^{3}}}ue^{-u^{2}/2}.
\end{eqnarray}%
Adler and Taylor \cite{TaylorAdler2009} write these components $\mathcal{M}_{l}([u,\infty ))=\frac{1}{\sqrt{2\pi }}%
H_{k}(u)e^{-u^{2}/2}$ and denote them Gaussian Minkowski functionals. The so-called "flag" coefficients are instead given by
\begin{equation}
\left[
\begin{array}{c}
i+l \\
l%
\end{array}%
\right] =\left(
\begin{array}{c}
i+l \\
l%
\end{array}%
\right) \frac{\omega _{i+l}}{\omega _{i}\omega _{l}}\text{ , for }\omega
_{i}=\frac{\pi ^{i/2}}{\Gamma (\frac{i}{2}+1)}\text{ ,}
\end{equation}%
that is, $\omega _{i}$ represents the area of the $i-$dimensional unit ball, $%
\omega _{1}=2,$ $\omega _{2}=\pi ,$ $\omega _{3}=\frac{4}{3}\pi.$
As a last ingredient, we write $\lambda$ for the parameter which represents the second derivative of
the covariance function at the origin.

We are now ready to present the general expression for the expected value of Lipschitz-Killing curvature, i.e., the Gaussian Kinematic Formula which reads (Theorem 13.2.1 in \cite{RFG}:

\begin{equation}
\lambda ^{i/2}\mathbb{E}\mathcal{L}_{i}(A_{u}(T(x),D))=\sum_{l=0}^{\dim
(D)-i}\left[
\begin{array}{c}
i+l \\
l%
\end{array}%
\right] \lambda ^{(i+l)/2}\rho _{l}(u)\mathcal{L}_{i+l}(D)\text{ .}
\label{GKF}
\end{equation}


%\subsection{Multipole fields}
As an application of the previous result, let us consider the Fourier components $\{f_\ell(\cdot)\}_{\ell=1,2,\dots}$ normalized to have variance one, the GKF yields immediately (compare
\cite{MarVad}, Corollary 5)%

\begin{equation}
\mathbb{E}\mathcal{L}_{0}(A_{u}(f_{\ell }(.),S^{2}))=2\left\{ 1-\Phi
(u)\right\} +\lambda_{\ell}\frac{ue^{-u^{2}/2}}{\sqrt{(2\pi )^{3}}}%
4\pi \text{ ;}  \label{sh1}
\end{equation}
\begin{equation}
\mathbb{E}\mathcal{L}_{1}(A_{u}(f_{\ell }(.),S^{2})) =\frac{\pi }{2}%
\lambda_{\ell}^{1/2}\frac{e^{-u^{2}/2}}{2\pi }%
4\pi=\pi \lambda_{\ell}^{1/2}e^{-u^{2}/2}\text{ ;} \label{sh2}
\end{equation}
and%
\begin{equation}
\mathbb{E}\mathcal{L}_{2}(A_{u}(T_{\ell }(.),S^{2}))=4\pi \times \left\{
1-\Phi (u)\right\} \text{ .}  \label{sh3}
\end{equation}

Of course, in order to exploit Lipschitz-Killing Curvatures/Minkowski functionals to implement data analysis tools the expected value by itself is not sufficient, but we need also analytic expression for the variance. The latter was derived in some recent results by \cite{CMW-EPC},....

For our purposes, the results in these papers can be summarized as follows; the asymptotic behaviour of each of the three Lipschitz-Killing Curvatures, evaluated on the excursion sets of random spherical harmonics, is dominated by a single, fully degenerate component, which can be written as:


\begin{equation*}
\mathtt{Proj}[\mathcal{L}_{k}(A_{u}(f_{\ell };\mathbb{S}^{2}))|2]
\end{equation*}%
\begin{equation}
=\frac{1}{2}\left[ 
\begin{array}{c}
2 \\ 
k%
\end{array}%
\right] \left\{ \frac{\lambda _{\ell }}{2}\right\}
^{(2-k)/2}H_{1}(u)H_{2-k}(u)\phi (u)\frac{1}{(2\pi )^{(2-k)/2}}\int_{\mathbb{%
S}^{2}}H_{2}(f_{\ell }(x))dx+a_{k}(\ell ),  \label{2GKF}
\end{equation}%
where
\begin{equation*}
a_{k}(\ell )=\left\{ 
\begin{array}{cc}
O_{p}(\ell ) & \text{for }k=0, \\ 
0 & \text{for }k=1,2%
\end{array}%
\right. \text{ .}
\end{equation*}%
It is important to notice that $\frac{\lambda _{\ell }}{2}=P_{\ell }^{\prime
}(1)$ represents the derivative of the covariance function of random
spherical harmonics at the origin, so that the term%
\begin{equation*}
\frac{\lambda _{\ell }}{2}\int_{\mathbb{S}^{2}}H_{2}(f_{\ell }(x))dx
\end{equation*}%
can be viewed as a (random) measure of the sphere induced by the Riemannian
metric (\ref{Riemetric}); recall indeed that for eigenfunctions $f_{\ell }$
on the sphere $\mathbb{S}^{2}$ the term ${\mathcal{L}}_{2}^{f_{\ell }}(%
\mathbb{S}^{2})$ which appears in (\ref{GKFA}) is exactly given by the area
of the sphere with radius $\left\{ \frac{\lambda _{\ell }}{2}\right\}
^{1/2}, $ i.e., 
\begin{equation*}
{\mathcal{L}}_{2}^{f_{\ell }}(\mathbb{S}^{2})=\frac{\lambda _{\ell }}{2}%
\times 4\pi =\frac{\lambda _{\ell }}{2}\int_{\mathbb{S}^{2}}H_{0}(f_{\ell
}(x))dx\text{ .}
\end{equation*}%
As was noted in ....., the Gaussian Kinematic Formula can be rewritten with a very similar expression to (\ref{2GKF}), i.e.: 
\begin{equation*}
\mathtt{Proj}[\mathcal{L}_{k}(A_{u}(f_{\ell };\mathbb{S}^{2}))|0]
\end{equation*}%
\begin{equation}
=\left[ 
\begin{array}{c}
2 \\ 
k%
\end{array}%
\right] \left\{ \frac{\lambda _{\ell }}{2}\right\} ^{(2-k)/2}H_{1-k}(u)\phi
(u)\frac{1}{(2\pi )^{(2-k)/2}}\int_{\mathbb{S}^{2}}H_{0}(f_{\ell
}(x))dx+b_{k}(\ell )\text{ ,}  \label{1GKF}
\end{equation}%
where 
\begin{equation*}
b_{k}(\ell )=\left\{ 
\begin{array}{cc}
2(1-\Phi (u))=O(1) & \text{for }k=0, \\ 
0 & \text{for }k=1,2%
\end{array}%
\right. .
\end{equation*}%
More explicitly (see \cite{DI}, \cite{MR2015}, \cite{ROSSI2015},.....), we have the following analytic expressions for the leading term components of the LKCs (expected values and dominant stochastic term):

\emph{a) Excursion Area} ($k=2$)%
\begin{equation*}
\mathtt{Proj}[\mathcal{L}_{2}(A_{u}(f_{\ell };\mathbb{S}^{2}))|0]=\left\{ 
\frac{\lambda _{\ell }}{2}\right\} ^{0}\left[ H_{-1}(u)\phi (u)\right] \int_{%
\mathbb{S}^{2}}H_{0}(f_{\ell }(x))dx\text{ ,}
\end{equation*}%
\begin{equation*}
\mathtt{Proj}[\mathcal{L}_{2}(A_{u}(f_{\ell };\mathbb{S}^{2}))|2]=\frac{1}{2}%
\left\{ \frac{\lambda _{\ell }}{2}\right\} ^{0}\left[ H_{0}(u)H_{1}(u)\phi
(u)\right] \int_{\mathbb{S}^{2}}H_{2}(f_{\ell }(x))dx\text{ ;}
\end{equation*}

\emph{b) (Half) Boundary Length} ($k=1$)%
\begin{equation*}
\mathtt{Proj}[\mathcal{L}_{1}(A_{u}(f_{\ell };\mathbb{S}^{2}))|0]=\left\{ 
\frac{\lambda _{\ell }}{2}\right\} ^{1/2}\sqrt{\frac{\pi }{8}}\left[
H_{0}(u)\phi (u)\right] \int_{\mathbb{S}^{2}}H_{0}(f_{\ell }(x))dx\text{ ,}
\end{equation*}%
\begin{equation*}
\mathtt{Proj}[\mathcal{L}_{1}(A_{u}(f_{\ell };\mathbb{S}^{2}))|2]=\frac{1}{2}%
\left\{ \frac{\lambda _{\ell }}{2}\right\} ^{1/2}\sqrt{\frac{\pi }{8}}\left[
H_{1}^{2}(u)\phi (u)\right] \int_{\mathbb{S}^{2}}H_{2}(f_{\ell }(x))dx\text{
;}
\end{equation*}

\emph{c) Euler-Poncar\'{e} Characteristic} ($k=0$)%
\begin{equation*}
\mathtt{Proj}[\mathcal{L}_{0}(A_{u}(f_{\ell };\mathbb{S}^{2}))|0]=\left\{ 
\frac{\lambda _{\ell }}{2}\right\} \left[ H_{1}(u)\phi (u)\right] \frac{1}{%
2\pi }\int_{\mathbb{S}^{2}}H_{0}(f_{\ell }(x))dx+2\left\{ 1-\Phi (u)\right\} 
\text{ ,}
\end{equation*}%
\begin{equation*}
\mathtt{Proj}[\mathcal{L}_{0}(A_{u}(f_{\ell };\mathbb{S}^{2}))|2]=\frac{1}{2}%
\left\{ \frac{\lambda _{\ell }}{2}\right\} \left[ H_{2}(u)H_{1}(u)\phi (u)%
\right] \frac{1}{2\pi }\int_{\mathbb{S}^{2}}H_{2}(f_{\ell }(x))dx+O_{p}(1)%
\text{ .}
\end{equation*}%



Given these results, \cite{cm1603} showed that the variances of LKCs
are dominated by the variance of the second order Wiener chaos. For
the Euler-Poincare characteristics the variance is given by
\begin{align*}
\text{Var}[\chi (A_{I}(f_{\ell }; {\cal S}^2) )]
&=\frac{\ell^3}{8 \pi } 
\Big[   \int_{I}  (-t^4+4 t^2 -1) e^{-\frac {t^2} 2 } d t \Big]^2+O(\ell^{5/2}),  
\end{align*}

\noindent and in particular for $I=[u,\infty)$

\begin{align*}
\text{Var}[\chi (A_{u}(f_{\ell }; {\cal S}^2) )]&= \frac{\ell^3}{8 \pi }  e^{- u^2  }   (u-u^3)^2  +O(\ell^{5/2})\\
&=\frac{\ell^3 }{4 } 
 \frac{e^{-u^2}}{2\pi} [H_3(u)+2H_1(u) ]^2  +O(\ell^{5/2}).
\end{align*}

The variances of LKCs converge to zero as the frequency increases,
so that fluctuations around expected values become negligible on
small scales, assuming the null assumptions hold. 

\section{Characterization of Critical Points for Random Spherical Harmonics}

As a further tool of investigation, we shall exploit in this paper also the behaviour of critical points for random spherical harmonics, which has recently been fully characterized by .... and ...., see also ....

More precisely, by definition critical points, extrema and saddle are given by, respectively:

\begin{equation*}
\mathcal{N}^{c}(f_\ell;u )=\mathcal{N}_{u}^{c}(f_{\ell })=\#\{x\in {\cal S}^2
:f_{\ell }(x)\geq u,\nabla f_{\ell }(x)=0\},
\end{equation*}
\begin{equation*}
\mathcal{N}^{e}(f_\ell;u)=\mathcal{N}_{u}^{e}(f_{\ell })=\#\{x\in {\cal S}^2
:f_{\ell }(x)\geq u,\nabla f_{\ell }(x)=0,\text{det}(\nabla ^{2}f_{\ell
}(x))>0\},
\end{equation*}
\begin{equation*}
\mathcal{N}^{s}(f_\ell;u)=\mathcal{N}_{u}^{s}(f_{\ell })=\#\{x\in {\cal S}^2
:f_{\ell }(x)\geq u,\nabla f_{\ell }(x)=0,\text{det}(\nabla ^{2}f_{\ell
}(x))<0\}.
\end{equation*}%
As evident, We used $a=c,e,s$ to label critical points, extrema and saddles respectively. \\

%\noindent It is now convenient to define the spherical harmonic empirical measure as
%follows: for all $z \in (-\infty, \infty)$,
%$$\Phi_{\ell}(z)=\int_{\cal{S}^2} \ind_{\{f_{\ell}(x) \le z\}} d z,$$
%where $\ind_{\{\cdot\}}$ is the indicator function. $\Phi_{\ell}(z)$
%provides the random measure of the set where the eigenfunctions lie
%below the value $z$.

We now recall the following results on the expectation and variance of these critical points:


For every interval $u \in \R$ we have, as $\ell \rightarrow \infty$,
\begin{equation*}
\mathbb{E}[\mathcal{N}_{u}^{a}(f_{\ell })] =\frac{2}{\sqrt{3}} \ell^2 \int_{u}^{\infty}\pi _{1}^{a}(t)dt+O(1),
\end{equation*}
where $a=c,e,s$ and for the density
functions

\begin{align}  
  \pi _{1}^{c}(t)&=\frac{\sqrt{3}}{\sqrt{8\pi
                   }}(2e^{-t^{2}}+t^{2}-1)e^{-\frac{                   
                   t^{2}}{2}}, \\
  \pi _{1}^{e}(t)&=\frac{\sqrt{3}}{\sqrt{2\pi
                   }}(e^{-t^{2}}+t^{2}-1)e^{-\frac{                   
                   t^{2}}{2}},\\
  \pi _{1}^{s}(t)&=\pi _{1}^{c}(t)-\pi
                   _{1}^{e}(t)=\frac{\sqrt{3}}{\sqrt{2\pi
                   }}e^{-\frac{3}{2}t^{2}}.\\
  \label{eqn:exp_crit}                                  
\end{align}



Similarly, for every $u \in \R$ as $\ell \rightarrow \infty $
\begin{equation*}
{\text{Var}}(\mathcal{N}_{u}^{a}(f_{\ell }))=\ell^3 \left[
\int_{u}^{\infty}p_{3}^{a}(t)dt\right] ^{2}+O(\ell ^{5/2}),
\end{equation*}
where,

\begin{align*}
  p_{3}^{c}(t)&=\frac{1}{ \sqrt{8 \pi }}e^{-\frac{3}{2} t^{2}
                }[2-6t^{2}-e^{t^{2}}(1-4t^{2}+t^{4})], \\
  p_{3}^{e}(t)&=\frac{1}{ \sqrt{8 \pi }}e^{-\frac{3}{2} t^{2}
                }[1-3t^{2}-e^{t^{2}}(1-4t^{2}+t^{4})],\\
  p_{3}^{s}(t)&=\frac{1}{ \sqrt{8 \pi }}(1-3t^{2})e^{-\frac{3}{2}%
                t^{2}}. \\
  \label{eqn:var_crit}
\end{align*}

The leading constants for the variances can be written more explicitly as
\begin{align}
  \left[ \int_{u}^{\infty }p_3^c(t)dt \right]^{2}&=\frac{1}{8 \pi} e^{- 3 u^2} u^2 (2+ e^{u^2}
                                                   (u^2-1))^2, \\
  \left[ \int_{u}^{\infty }p_3^e(t)dt \right]^{2}&= \frac{1}{8 \pi}  e^{- 3
                                                   u^2} u^2 (1+ e^{u^2} (u^2-1))^2,\\
  \left[ \int_{u}^{\infty }p_3^s(t)dt \right]^{2}&=\frac{1}{8 \pi}  e^{- 3 u^2} u^2.
\end{align}

It is important to stress how the leading terms in the variances cancel in all cases at the threshold $u=0$. This is a form of the so-called "Berry's cancellation phenomenon", which holds also for the Lipschitz-Killing Curvatures; it was conjectured for boundary length in Berry..., and it is discussed at length in .... and ....


\section{Numerical results}\label{sec:numerical}
In this section we describe the comparison of the analytical
results outlined in the previous sections to the corresponding
results from simulations.  In all cases we generated 100 map
realizations of an input power spectrum using the
\healpix\cite{healpix} package. We estimated LKCs from each
simulations and compared their mean with the analytical results.
We found an excellent agreement in all the cases that we
investigated; more precisely, not only the estimated curves are
always well within the $68\%$ Confidence Interval (CL), but
actually as shown below they are for practical purposes basically
indistinguishable from the theoretical predictions even with a
relatively low number of Monte Carlo simulations.



\begin{figure} %[H]
\begin{center}
  \includegraphics[width=0.32\textwidth,angle=0]{\figdir{rms_diff_proj2_ell150_genus.pdf}}
  \includegraphics[width=0.32\textwidth,angle=0]{\figdir{rms_diff_proj2_ell150_length.pdf}}
  \includegraphics[width=0.32\textwidth,angle=0]{\figdir{rms_diff_proj2_ell150_area.pdf}}
  \includegraphics[width=0.32\textwidth,angle=0]{\figdir{rms_diff_proj2_ell300_genus.pdf}}
  \includegraphics[width=0.32\textwidth,angle=0]{\figdir{rms_diff_proj2_ell300_length.pdf}}
\includegraphics[width=0.32\textwidth,angle=0]{\figdir{rms_diff_proj2_ell300_area.pdf}}    
\caption{Variance of the difference between theory and simulation:
  First order (black and grey) vs
  Second order (red) subtracted MFs. The rows from left to right are
  Euler-poicare characteristics, perimeter length and area,
  respectively. The legend shows the multipoles at
  which the LKCs are evaluated: the upper panel is for $\ell=200$ and
  the lower panel is for $\ell=400$. Grey Shades are $68, 95$ and $99 \%$
  percentiles estimated from 100 simulations. \label{fig:m1ell}}
\end{center}
\end{figure}


\begin{figure} %[H]
\begin{center}
  \includegraphics[width=0.32\textwidth,angle=0]{\figdir{var_proj2_vs_data_ell150_genus.pdf}}
  \includegraphics[width=0.32\textwidth,angle=0]{\figdir{var_proj2_vs_data_ell150_length.pdf}}
  \includegraphics[width=0.32\textwidth,angle=0]{\figdir{var_proj2_vs_data_ell150_area.pdf}}
  \includegraphics[width=0.32\textwidth,angle=0]{\figdir{var_proj2_vs_data_ell300_genus.pdf}}
  \includegraphics[width=0.32\textwidth,angle=0]{\figdir{var_proj2_vs_data_ell300_length.pdf}}
\includegraphics[width=0.32\textwidth,angle=0]{\figdir{var_proj2_vs_data_ell300_area.pdf}}    
\caption{Comparison of Variance of MFs from theory and simulation:
  First order (black and grey) vs
  Second order (red) subtracted MFs. The rows from left to right are
  Euler-Poincare' characteristics, perimeter length and area,
  respectively. The legend shows the multipoles at
  which the LKCs are evaluated: the upper panel is for $\ell=200$ and
  the lower panel is for $\ell=400$. Grey Shades are $68, 95$ and $99 \%$
  percentiles estimated from 100 simulations. \label{fig:m1ell}}
\end{center}
\end{figure}




\begin{figure} %[H]
\begin{center}
  \includegraphics[width=0.32\textwidth,angle=0]{\figdir{critical_pdf_mpow1_meann128.pdf}}
  \includegraphics[width=0.32\textwidth,angle=0]{\figdir{extrema_pdf_mpow1_meann128.pdf}}
  \includegraphics[width=0.32\textwidth,angle=0]{\figdir{saddle_pdf_mpow1_meann128.pdf}}  
\caption{Comparison of expectation PDF of critical, extrema, and saddle points from theory and simulation:
  The legend shows the multipoles at
  which the simulations curves are evaluated: Grey Shades are $68, 95$ and $99 \%$
  percentiles estimated from 100 simulations. \label{fig:exp_crit}}
\end{center}
\end{figure}


\begin{figure} %[H]
\begin{center}
  \includegraphics[width=0.32\textwidth,angle=0]{\figdir{variance_critical_pdf_num_vs_theory.pdf}}
  \includegraphics[width=0.32\textwidth,angle=0]{\figdir{variance_extrema_pdf_num_vs_theory.pdf}}
  \includegraphics[width=0.32\textwidth,angle=0]{\figdir{variance_saddle_pdf_num_vs_theory.pdf}}  
\caption{Comparison of Variance of critical, extrema, and saddle points from theory and simulation:
  The legend shows the multipoles at
  which the simulation curves are evaluated: Grey Shades are $68, 95$ and $99 \%$
  percentiles estimated from 100 simulations. \label{fig:var_crit}}
\end{center}
\end{figure}




\subsection*{Simulations and Algorithm}
To test the validity of equations presented above, we used the \healpix \emph{synfast} to simulate a Gaussian realization map the best-fit Planck power spectrum. Note, however, that this particular choice of the power spectrum has no influence on the results we shall provide.  

For testing the Gaussianity of the Planck full-sky map, we used the 2018 Planck data and the corresponding FFP10 CMB only simulations. It is well known that the Planck full-sky map release should not be used for a serious cosmological analysis as part of the data is known to be highly unreliable and a confidence mask should be used. Our results derived from these maps is mostly to show a proof of concept to the use case of the MFs and critical points results derived using the second order Gaussian Kinematic Formula.    

For the case of the Gaussian simulations, a single multipole map is
obtained by taking the corresponding spherical harmonic coefficient
and applying the \healpix \emph{alm2map}. For all the other cases
where we start our analysis from existing maps, we first perform a
\emph{map2alm} operation to obtain 
the spherical harmonic coefficients of the map. We followed a similar
procedure as we did for the Gaussian simulation case to obtain
multipole, $T_\ell(x)$, maps.
In all cases the maximum multipole is set to twice the map resolution
parameter $\nside$.

These maps are then normalized by the corresponding sample root mean square, which is computed from the map itself. This is necessary as we have shown above using the sample rms minimizes MFs estimator variance.

We compute  the three Minkowski Functionals, which are equivalent to the LKCs up to constant factors, and critical point counts from these normalized multipole maps.  

The MFs are obtained using a Fortran implementation of the algorithms described in
Appendix G of \cite{Gay2012_NGPeaks}. In short, these algorithms can be described
as follows: the area, i.e. the first MF, is computed by evaluating
the number of pixels above a certain threshold. The perimeter length, the
second MF, is computed by tracing isocontour lines in pixel space.
For a sufficiently high-resolution map, pixels around isocontour
lines have different signs relative to the contour line, after
normalizing the lines to zero. To measure the length of these
lines, sets of four pixels are compared; when at least two of them
have different signs, the locations where the contour line enters
and exits these sets of pixels are determined and the length is
iteratively calculated by standard dot product. 

The Euler-Poincare\'{e}, the third MF, is computed by means of its
characterizations through Morse theory; more explicitly, critical
points are determined as the pixels where the gradient vanishes.
The Hessian matrices around these critical points are computed,
and their so-called indexes (i.e., the sign of their determinant,
or the product of their eigenvalues) are evaluated. Positive
indexes correspond to extrema (minima plus maxima), negative
indexes to saddles; in two dimensions, the Euler-Poincar\'{e}
characteristic is simply obtained as the difference between the
number of extrema and the number of saddles. 

Our detailed investigation using different algorithms to compute the
Euler-Poincare\'{e} characteristic showed that for a map defined at a
given $N_{\rm side}$, the maximum multipole for which a percent
numerical accuracy can be obtained is $\ell_{max} \sim N_{\rm
  side}/3$. Moreover, since a polynomial transformation of a band
limited map increases the bandwidth accordingly, in the following we
show only results for multipoles $\ell<500$. While it would be
possible to cover larger values, we do not believe this is
essential for our purpose in this paper.




\subsection*{Results: Gaussian fields}

In \figref{fig:m1ell} we compare the multipole space analytical
results (red curve) given in \secref{ssec:lkc_gauss} with that of
the simulations (black curve - mean of the simulations). The
$68\%,95\%$ and $99\%$ CLs are shown from dark to light grey
bounds. From left to right panels, the plots shows the results
corresponding to multipoles $\ell=5,50,105$. We stress that our
fit is extremely accurate, even at very low multipole values where
the flat-sky approximation which is usually adopted cannot be
expected to hold.  We also note the improved concentration around
the expected values at higher-multipoles; indeed, the same
behaviour of these variances can be predicted analytically, but we
delay these results for future work.

Likewise, \figref{fig:m1beta} shows analogous results in needlet
space; the colors for different curves have the same meaning as
described above. The displayed results cover the frequencies
$j=10,12,14$  which for $B=1.5$  correspond to multipoles in the order of
60,130,200; these results are even more accurate than in the
multipole case, in particular the decay of Cosmic Variance is
faster.


\section{Summary and Conclusion}

In this paper, we illustrated a number of applications for
Cosmological data analysis of the GKF, (see \cite
{TaylorAdler2003, Taylor2006, TaylorAdler2009},
\cite{adlerstflour}, \cite{RFG}). The GKF allows to evaluate exact
expected values for Lipschitz-Killing curvatures (Minkowski
functionals) in a number of circumstances of applied interest,
covering in particular full-sky experiments (accounting for the
geometry of the sphere).

We used the GKF on random fields derived
by harmonic transforms, allowing for the further advantage
of better control of Cosmic Variance effects and localization.  In
particular we provided the analytic expressions for the second order Minkowski
functionals of single multipole fields.  All
the results reported are validated by an extensive Monte Carlo study,
which demonstrates an extremely good agreement between predictions and
simulations.



\section{Acknowledgments}
The authors acknowledge support from ERC Grant 277742 Pascal.
We acknowledge the use of resources from the
Norwegian national super-computing facilities, NOTUR. Maps and results
have been derived using the \healpix (http://healpix.jpl.nasa.gov)
software package developed by \cite{healpix}.


\bibliography{mf_literature_wabstract,Planck_bib}
%\bibliography{gkf_proj2}



\end{document}

%%% Local Variables:
%%% mode: latex
%%% TeX-master: t
%%% End:
